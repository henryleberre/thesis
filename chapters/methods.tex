\chapter{Methods}
\label{ch:methods}

MFC with chemistry (herein CheMFC), is based on a 1-fluid formulation of the 5-equation
model from Allaire et al. \cite{ALLAIRE2002577}, augmented with $N_{sp}$ species
transport equations from Poinsot and Veynante \cite{poinsot:hal-00270731}.
%
\begin{align*}
    \frac{\partial\left(\rho Y_k\right)}{\partial t} & = \dot\omega_k - \nabla \cdot (\rho Y_k \Vec{u}) - \nabla \cdot (\rho Y_k \Vec{V_k})
\end{align*}

We naturally take $Y_k$ as our primitive variable and $\rho Y_k$ as our conservative variable
for species $k \in \left\{1,\dots,N_{sp}\right\}$. Each $Y_k$ is WENO-reconstructed
and the associated advection flux computed using either the HLL or HLLC Riemann solver.
Therein, $\gamma$, the ratio of specific heat capacities, is computed in one of two ways:
%
\begin{align*}
    \gamma &= \frac{c_p}{c_v} = \frac{\sum\limits_{k=1}^{N_{sp}} Y_k c_{p,k}}{\sum\limits_{k=1}^{N_{sp}} Y_k c_{v,k}} & \text{or} & & \gamma &= 1 + \left(\sum\limits_{k=1}^{N_{sp}} \frac{X_k}{\gamma_k - 1} \right)^{-1}
\end{align*}
%
the latter formulation being taken from Deiterding \cite{soton380602}.

To compute temperature ($T$) from the mixture density ($\rho$) and our conservative energy
variable ($\rho E$), we must perform an iterative solve using the Newton-Raphson method,
as to satisfy
\begin{align}\label{eq:energy_newton_pyro}
    e - \sum\limits_{k=1}^{N_{sp}} e_k\left(T\right) Y_k = 0
\end{align}
where, $e$, the internal mixture energy, can be defined as $e = E - \frac{1}{2} \norm{u}^2$.
Pressure is simplistically defined using the ideal gas law, $p = \rho R T$
where $R$ is the ideal gas constant.

We extended Pyrometheus \cite{Pyrometheus2024} to provide a Fortran interface
for computing the reaction source term ($\dot\omega_k$) and the specific heat capacities
at constant volume and pressure, as well as performing the
Newton-Raphson solve for temperature described in \autoref{eq:energy_newton_pyro},
while supporting OpenACC offload. MFC's build system was amended to allow for
mechanism-specific code-generation and compilation to occur transparently to the user.

To conclude, our conservative and primitive variable vectors are defined as:
%
\begin{align*}
    \Vec{Q}_{\text{cons}} & = \left[\begin{array}{c}
        \alpha\rho \\
        \rho u_i \\
        \vdots \\
        \rho u_{N_{dim}} \\
        \rho E \\
        \alpha \\
        \rho Y_1 \\
        \vdots \\
        \rho Y_{N_{sp}} \\
        T
    \end{array}\right]
    &
    \Vec{Q}_{\text{prim}} & = \left[\begin{array}{c}
        \alpha\rho \\
        u_i \\
        \vdots \\
        u_{N_{dim}} \\
        p \\
        \alpha \\
        Y_1 \\
        \vdots \\
        Y_{N_{sp}} \\
        T
    \end{array}\right]
\end{align*}
%
where the fluid void fraction ($\alpha$) and temperature ($T$) variables,
and their respective equations, could be done without, but kept for simplicity,
given $\alpha$ is a remnant from the 5-equation model (given our 1-fluid formulation)
and temperature, derived from energy, is cached (and not recomputed) every time it is used.
The memory and computational overhead is considered negligible since the species transport
equations are expected to dominate the number of equations in the system.