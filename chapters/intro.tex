\chapter{Introduction}
\label{ch:introduction}

Combustion processes power the modern world, from cooking appliances, aircraft,
and cars, to industrial manufacturing equipment, rockets, and missiles. About 35\%
of the world's energy is still produced by the exothermic reaction of coal-burning.

Scientists have developed increasingly sophisticated models to describe the mechanics
of combustion in the presence of fluid flow, where many of its applications lie.
However, these models most often only correctly describe local behavior, and cannot
easily be extended to predict how systems of non-trivial size and complexity will
evolve over time.

In some cases, simplifying assumptions can be made and used alongside
empirical results to reason about plausible outcomes, though the confidence with
which these predictions can be made is variable. In other cases, when our predictive
power is insufficient and the consequences of failure are severe -- as may be the case
when designing critical infrastructure -- we must often rely on computer simulations
and experiments, to ground our designs and decisions thereabout in objectivity.

To illustrate, engineers developing the Apollo program's F-1 rocket engine famously
struggled with combustion instability, a phenomenon that remains a challenge today
and a focus of study. Small local disturbances in the combustion chamber would
lead to catastrophic failures. Fortunately, this problem manifested itself during
testing and mitigation strategies were developed. Undiagnosed, however, difficult
to predict issues like these could lead to the loss of human life. We are therefore
motivated to deepen our understanding of combustion processes.

Our goal in this work is to begin the development of a high-performance, open-source,
and permissibly licensed (MIT) solver capable of simulating reacting flows.
We hope to help engineers and researchers build novel, efficient, reliable, and
safe combustion systems without proprietary software they cannot improve or share
with others. To this end, we are extending the Multi-Component Flow Code (MFC),
a GPU-accelerated multi-physics compressible fluid flow solver capable of scaling
to the largest supercomputers in the world today.
