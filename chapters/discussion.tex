\chapter{Discussion and Future Work}

We set out to extend the Multi-component Flow Code (MFC) with combustion modeling
capabilities, in an effort to provide a performant and permissively licensed solution
for combustion simulation in the open-source space.
We chose MFC for its performance, scalability, cross-compatibility across GPU
vendors and HPC systems, interface-capturing schemes, and extensibility.

The goal was met with MFC's v4.9.7 release, the first to contain validated 0-1D
combustion modeling capabilities, as seen in \autoref{ch:validation}. Some fixes
required for the 2D simulation shown in \autoref{ch:example} are awaiting to be
merged.

However, several limitations remain, preventing us from simulating some more
complex scenarios. Three main areas of improvement are identified: the lack of
chemical diffusion, the need for non-ideal equations of state, and the need for
varying boundary conditions along the edges of the domain. The first two are
critical for the fidelity of the simulations, while the third is necessary to
properly define the associated problem setups, such as having an inflow boundary
condition between two no-slip walls.

Performance and high memory footprint will remain a concern, even when offloading
to GPUs, given the number of equations to be solved is an order of magnitude higher
than in regular MFC simulations. It impacts the size of the simulations we can
reasonably expect to run. This problem, however, is to be expected for this class
of solver, and one is we can attempt to further mitigate by optimizing the
Pyrometheus \cite{Pyrometheus2024} thermochemistry library.