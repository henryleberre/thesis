\chapter{Literature Review}
\label{ch:lit}

In this work, we will continue to develop the Multi-Component Flow Code (MFC) \cite{Bryngelson_2021},
the flow solver mentioned in the introduction, born at the Computational and Data-Driven
Fluid Dynamics research group at the California Institute of Technology. MFC is as a
software package for the simulation of multi-component
and multi-phase flows, targeted at solving problems in the defense, aerospace, and
healthcare sectors. For many problems of interest, such as the modeling
of bubble cavitation in the wake of a submarine's propeller, computer simulations
help to overcome the burdens of physical experimentation and theoretical analysis, while
allowing for more flexibility in the design and validation of new technologies.

Fully resolved simulations require significant computational resources, surpassing
the capabilities of a single computer. High-performance computing (HPC) systems
provide a network of computers that can be used simultaneously to solve a single
problem. To this end, MFC's Fortran codebase leverages the Message Passing Interface (MPI) to
distribute work among multiple computer nodes, assigning portions of the computational
domain to individual processors in the pre-processing step.

Communication between processors
is required to compute fluxes across the faces of cells residing on the boundary
of processor domains, as they lack information about their neighbors.
MFC makes use of "halo" regions, thin layers of cells surrounding the
interior domain of each processor, whose purpose is to store the values of the neighboring
processors' boundary cells, for use in the computation of fluxes. These regions
are updated every stage of our Runge--Kutta time integration scheme. At the very edge
of the computational domain, non-periodic user-specified boundary conditions are
enforced by prescribing the values of the halo cells thereon.

In a subsequent
research effort, we introduced hardware acceleration through the use of General-Purpose
Graphics Processing Units (GPGPUs) \cite{radhakrishnan2024method}. At the time, on the nation's
most most powerful supercomputer, Oak Ridge Leadership Computing Facility's (OLCF) Summit,
a $40$-times node-wise speedup was achieved when comparing GPU-offload to traditional
CPU computation, greatly reducing the cost of running large simulations.
Offloading was done through the OpenACC programming model, using meta-programming
and case-specific optimizations to achieve maximum performance, among other techniques.

MFC employs the 4- and 5-equation models from Preston et al. \cite{ALLAIRE2002577} to model the physics
of flow. These are systems of partial differential equations of the form:
%
\begin{equation}
    \frac{\partial q}{\partial t} + \nabla \cdot F(q) + h(q)\nabla\cdot u = s(q)
\end{equation}

\noindent where $q$ is the state vector of conserved quantities, $F$ the flux tensor,
$u$ the velocity field, and $s(q)$ the associated source term \cite{Bryngelson_2021}.
For the purposes of this review, we will not detail the $h(q)\nabla\cdot u$ term.
Each component of $q$ represents a conserved quantity at any given grid location,
such as $\rho E$ (density times energy). Fluxes designated by $F(q)$ 
represent how each component in the state vector is expected to be transported
to and from the current location (to and from other cells). By definition, a flux
represents the net transport of a quantity through a face. MFC is a finite-volume
code and represents the domain as a collection of cells with non-zero volume \cite{Bryngelson_2021}.
Naturally, adjacent cells share faces. Preston et al. \cite{ALLAIRE2002577} provide
expressions for $q$, $F(q)$ ,$h(q)$, and $s(q)$, capable of modeling the flows MFC
is interested in simulating.

To compute fluxes across cell faces, MFC uses (Monotonicity Preserving) WENO
(Weighted Essentially Non-Oscillatory) schemes to reconstruct values at cell
faces from their cell-centered counterparts in a way that does not introduce
spurious oscillations into the simulation \cite{CORALIC201495,JOHNSEN2006715}.
Indeed, the simulation stores and manipulates state vectors at the cell centers,
not their faces, whose values are necessary to calculate derivatives using finite
differences and other numerical methods. WENO uses the cell-centered values in the
neighborhood of a cell to derive its face-centered values through weighted
interpolation, while attempting to minimize numerical artifacts that can result
in the formation of spurious oscillations. WENO schemes are dissipative, meaning
they can dampen oscillations and smear discontinuities.

From these WENO-reconstructed face-centered values, one could naively compute
fluxes across cell faces using $F(q)$. However, MFC is interested in simulating
flows that involve shocks, often in the form of propagating waves. They give rise
to sharp discontinuities in the flow field, where across two cell faces, the
value of a state variable can abruptly jump from one value to another. For example,
if two fluids are inserted, each on one side of the domain with their own density,
an interface forms between the two components, where density jumps from one value
to another without a smooth transition. Differentiating a state variable at the
discontinuity can lead to unphysical solutions, as the field can no longer be
modeled as continuous. This is due to the discreteness of the grid imposed by
computer simulation. Approximate Riemann solvers (solving the Riemann Problem)
such as HLLC (Harten-Lax-van Leer-Contact) are used to determine the flux through
a discontinuity \cite{toro2009riemann}. They use information about the shock waves
moving through the domain to determine fluxes accurately.

The methods presented above let us determine $\frac{\partial q}{\partial t}$ at
a particular time-step, the (partial) time derivative of the conservative state
vector. These are called "right-hand-side (RHS) evaluations". To advance the
simulation through time, we must integrate it -- with respect to time. MFC uses a popular class of
iterative schemes known as explicit Runge--Kutta (RK) methods \cite{BUTCHER1996247}.
They evaluate and compose $\frac{\partial q}{\partial t}$ multiple times per
time-step to provide a guess for the $q$ state vector at the next time-step.
Their popularity lies in their great accuracy at a low computational cost
(measured in the number of right-hand-side evaluations). There are multiple versions
of this method. For example, RK2 can be written as

$$
\begin{cases}
    K_1 &= h f\left(t_n, q_n\right) \\
    K_2 &= h f\left(t_n + \frac{h}{2}, q_n + \frac{K_1}{2}\right) \\
    q_{n+1} &= q_n + K_2 \\
    t_{n+1} &= t_n + h
\end{cases}
$$

\noindent for a given (time) step size $h \ll 1$ if we express our problem as
$\frac{\partial q}{\partial t}=f(t,q)$ \cite{BUTCHER1996247}. MFC simulations
often make use of another variation, a third-order total variation diminishing (TVD)
Runge--Kutta method from Gottlieb and Shu \cite{70473541-9e24-3684-8ded-69b842aad3b3}
that provides more guarantees about the stability of the integration scheme.

To simulate chemistry, more equations and state variables are needed. Poinsot
and Veynante provide a framework for simulating chemical advection, diffusion,
and chemical kinetics, among other phenomena \cite{poinsot:hal-00270731}. Most notably,
they provide the following conservative transport equation for species $k$:
\begin{equation}
    \frac{\partial\left(\rho Y_k\right)}{\partial t} = \dot\omega_k - \nabla \cdot (\rho Y_k \Vec{u}) - \nabla \cdot (\rho Y_k \Vec{V_k})
\end{equation}

\noindent where $Y_k$ is the mass fraction of the $k$-th species, $\rho$ is the density, $u$ is the velocity field,
$\dot\omega_k$ is the reaction source term for species $k$, and $\Vec{V_k}$ is the correction velocity for species $k$
(which accounts for chemical diffusion). As is evident from the above equation, the authors recommend the use $q_k=\rho Y_k$
as the conservative variable associated with species $k$ and, $Y_k$ (or $X_k$, the molar fraction) as the associated
primitive variable.

Martínez-Ferrer et al. \cite{MARTINEZFERRER201488} present detailed
validation procedures for combustion CFD codes, some of which we will follow in
\autoref{ch:validation}. These mostly stress advection fluxes, the reaction
source term, shock-capturing, and our temperature and pressure calculations.
The two one-dimensional shock tube problems they present, \textit{Multicomponent inert shock tube}
and \textit{Multi-species reactive shock tube} are already present in MFC's source package
but without the chemistry.

To validate the chemical diffusion term, we can either begin with the full formulation
shown above, or consider a simpler case such as binary diffusion, where only two
species are present and susceptible to diffuse into each other. With these simplifying
assumptions, Poinsot and Veynante \cite{poinsot:hal-00270731} show that the
transport equation can be rewritten as
%
$$
\frac{\partial\left( \rho Y_k\right)}{\partial t} = \dot\omega_k - \nabla \cdot \left( \rho Y_k \right) - \nabla \cdot \left( \rho D \nabla Y_k \right)
$$
%
\noindent where $D$ is a constant specific to the two species mixing, dropping the need to
solve a system for diffusion velocities.

This can be validated using results from Antonio~L.~Sánchez~et~al., who study
one-dimensional binary mixing problems \cite{10.1063/1.2221349}. Their work lets
us validate our implementation of the above differential equation with the
following problem definition and time-dependent exact analytical solution thereof
%
$$
\begin{cases}
    \rho\left(r,t\right) &= \rho_b + \frac{\rho_a-\rho_b}{2} \left[1-\text{erf}\left(\frac{r}{2\sqrt{Dt}}\right)\right] \\
    Y_1(r,t) &= \frac{\rho_a\left[\rho_b-\rho(r,t)\right]}{\rho(r,t)\left[\rho_b-\rho_a\right]} \\
    Y_2(r,t) &= 1 - Y_1(r,t)
\end{cases}
$$

\noindent where $t$ is the time, $\rho_a$ and $\rho_b$ are, respectively, the
densities of the two mixing gases, $\rho\left(r,t\right)$ and $Y_k(r,t)$ are,
respectively, the density and mass fraction of the $k$-th species at time $t$
and position $r$ along the 1D domain. The mixing layer has a thickness of $2\sqrt{Dt}$
(in domain space).

Similarly, multi-species diffusion can first be validated using the above binary
diffusion case, before moving to the more complex case described by Martínez-Ferrer et al. \cite{MARTINEZFERRER201488}.

Multiple software packages exist to assist in computing correction velocities
and the reaction source term such as Cantera \cite{cantera}, Pyrometheus \cite{Pyrometheus2024},
and KinetiX \cite{danciu2024kinetixperformanceportablecode}. Pyrometheus and
KinetiX are both designed to generate optimized thermochemistry routines from
Cantera mechanisms, aiming to maximize performance by unrolling loops, optimizing
the computation graph, and hard-coding
mechanism constants like NASA polynomial coefficients, so as to reduce
memory accesses and eliminate redundant computations. Neither provide Fortran bindings.

One application of simulating reacting flows is the design and validation of scramjets.
They are engines whose combustion happens at supersonic speeds (Mach $>$ 1)
\cite{doi:10.2514/6.2009-127}. F. Ladeinde brings attention to spray modeling,
encouraging more research to be done in the area, and highlights the prevalence of
flamelet methods as a tool to simplify combustion simulation \cite{doi:10.2514/6.2009-127}.
This method is also presented in Poinsot and Veynante's book \cite{poinsot:hal-00270731}
as a simplification that models each element of the frame-front (the part of the flame
that is reacting) as a laminar flame. These elements are called flamelets and can
provide significant performance improvements \cite{poinsot:hal-00270731,doi:10.2514/6.2009-127}.
